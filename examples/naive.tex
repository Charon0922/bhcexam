% 使用 BHCexam 文档类,并传递选项
\documentclass[answers]{BHCexam}
\usepackage{hyperref}

\begin{document}

% 第一行主标题
\title{BHCexam试卷排版宏包}

% 第二行主标题
\subtitle{样例}

% 考试说明
\notice{满分100分, 10分钟完成.}

% 命题人信息
\author{微信关注公众号:橘子数学}

% 考试日期
\date{2019.12.1}

% 生成试卷头
\maketitle

\begin{groups}

% 第一个题组,显示分值,不预留空间
\group{填空}{本题组共1小题,共30.0分}
\begin{questions}[s]

% 填空题,两个空
\question[30] 橘子数学的网址是\key{www.mathcrowd.cn}, 微信服务号\key{橘子数学}.
\question[30] 橘子数学趣味挑战的网址是\key{qa.mathcrowd.cn},微信订阅号是\key{试题工坊}, \key{橘子数学题库}.

\end{questions}

% 第二个题组,显示分值,不预留空间
\group{选择}{本题组共2小题,共40.0分}
\begin{questions}[ps]

% 选择题,四个选项
\question[30] 以下哪一项不是橘子数学社区的宗旨\key{C}.
\fourchoices{开放}{高效}{无视版权}{合作}

% 解答,4cm 参数被忽略
\begin{solution}{4cm}
\method 橘子数学社区的宗旨是开放、高效、合作、变革.
\method 见 \url{http://docs.mathcrowd.cn/zh_CN/latest/community/principles.html}
\end{solution}

% 选择题,五个选项
\question[40] 以下数学公式显示有明显瑕疵的是\key{D}.
\fivechoices{$\sin A$}{$2+3\mathrm{i}$}{$x^2$}{$\ln x$}{$\mathrm{e}^{\mathrm{i}\theta}$}

\begin{solution}{4cm}
\methodonly D 中正确的公式显示效果为$\ln{x}$.
\end{solution}
\end{questions}

% 第三个题组,显示分值,预留空间
\group{主观题}{本题组共1小题,共30.0分}
\begin{questions}[st]
% 简答题,两个小问
\question[30] 请回答以下问题:
\begin{subquestions}
    \subquestion 你觉得有必要创建这样一个试题社区吗? 为什么?
    \subquestion 你对社区的建设有什么建议.
\end{subquestions}

% 解答,学生版会预留8cm的答题空间.
\begin{solution}{4cm}
	\methodonly 欢迎加入用户群组发言讨论. 
	
telegram 交流群组: https://t.me/mathcrowd

QQ 群: 319701002

Github项目页: \url{https://github.com/mathedu4all/bhcexam}

\score{30}{30}
	
\end{solution}
\end{questions}

\end{groups}
\end{document}
