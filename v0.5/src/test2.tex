\documentclass[16kpaper]{BHCexam}
\begin{document}

\maketitle
\mininotice

\begin{questions}
\tiankong
\question 已知~$\bm{a}=(k,-9)$、$\bm{b}=(-1,k)$, $\bm{a}$~与~$\bm{b}$~为平行向量,
    则~$k=$\key{$\pm3$}.

\question 若函数~$f(x)=x^{6m^2-5m-4}\,(m\in\mathbb{Z})$~的图像关于~$y$~轴对称,
    且~$f(2)<f(6)$, 则~$f(x)$~的解析式为\key{$f(x)=x^{-4}$}.

\question 若~$f(x+1)=x^2\,(x\leq0)$, 则~$f^{-1}(1)=$\key{0}.

\question 在~$b\text{g}$~糖水中含糖~$a\text{g}$\,($b>a>0$), 若再添加~$m\text{g}$~糖~($m>0$),


\question 已知~$f(x)=1-\textbf{c}_8^1x+\textbf{c}_8^2x^2-\textbf{c}_8^3x^3+\cdots+\textbf{c}_8^8x^8$,
    则~$f\big(\dfrac{1}{2}+\dfrac{\sqrt{3}}{2}\textbf{i}\big)$~的值是\key{$-\dfrac{1}{2}-\dfrac{\sqrt{3}}{2}\textbf{i}$}.

\question 自然数~$1,2,3,\ldots,10$~的方差记为~$\sigma^2$,
    其中的偶数~$2,4,6,8,10$~的方差记为~$\sigma_1^2$,
    则~$\sigma^2$~与~$\sigma_1^2$~的大小关系为~$\sigma^2$\key{$>$}$\sigma_1^2$.

\question 若~$\theta$~为三角形的一个内角, 且~$\sin\theta+\cos\theta=\dfrac{2}{3}$,
    则方程~$x^2\csc\theta-y^2\sec\theta=1$~表示的曲线的焦点坐标是\key{$\big(\pm\dfrac{\sqrt{6}}{3},0\big)$}.


\question 高为~$h$~的棱锥被平行于棱锥底面的截得棱台侧面积是
    原棱锥的侧面积的~$\dfrac{5}{9}$,
    则截得的棱台的体积与原棱锥的体积之比是\key{$19:27$}.

\question 以椭圆~$\dfrac{x^2}{169}+\dfrac{y^2}{144}=1$~的右焦点为圆心,
    且与双曲线~$\dfrac{x^2}{9}-\dfrac{y^2}{16}=1$~的渐近线相切的圆方程是\key{$(x-5)^2+y^2=16$}.


\question 若~$\sqrt{\,\sin x}$~是有理数且~$x$~不是~$\dfrac{\pi}{6}$~的整数倍,
    则~$x$~可能取的值是\key{$\arcsin\dfrac{1}{4}$ 等}.(只要求写出一个)

\question 马路上有编号~1~到~10~的~10~盏路灯, 为节约用电又不影响照明,
    可以关掉其中的~3~盏, 但又不能同时关掉相邻的两盏, 也不能关掉两端的路灯,
    满足条件的关灯方法有\key{$20$}种.


\question 以椭圆~$\dfrac{x^2}{169}+\dfrac{y^2}{144}=1$~的右焦点为圆心,
    且与双曲线~$\dfrac{x^2}{9}-\dfrac{y^2}{16}=1$~的渐近线相切的圆方程是\key{$(x-5)^2+y^2=16$}.

\question 若~$\sqrt{\,\sin x}$~是有理数且~$x$~不是~$\dfrac{\pi}{6}$~的整数倍,
    则~$x$~可能取的值是\key{$\arcsin\dfrac{1}{4}$ 等}.(只要求写出一个)

\question 马路上有编号~1~到~10~的~10~盏路灯, 为节约用电又不影响照明,
    可以关掉其中的~3~盏, 但又不能同时关掉相邻的两盏, 也不能关掉两端的路灯,
    满足条件的关灯方法有\key{$20$}种.


\question 以椭圆~$\dfrac{x^2}{169}+\dfrac{y^2}{144}=1$~的右焦点为圆心,
    且与双曲线~$\dfrac{x^2}{9}-\dfrac{y^2}{16}=1$~的渐近线相切的圆方程是\key{$(x-5)^2+y^2=16$}.

\newpage

\xuanze
\question 已知集合~$A=\{x\mid \abs{x-1}<3 \}$,
集合~$B=\{y| y=x^2+2x+1,x\in\mathbb{R}\}$, 则~$A\cap
\complement_U B$~为\key{C}.
\fourchoices{$[\,0,4)$}{$(-\infty,-2\,]\cup[4,+\infty)$}{$(-2,0)$}{$(0,4)$}

\question 若~$a$、$b$~是直线, $\alpha$、$\beta$~是平面,
则以下命题中真命题是\key{D}.\\
\fourchoices{若~$a$、$b$~异面, $a\subset\alpha$,$b\subset\beta$, 且~$a\perp b$, 则~$\alpha\perp\beta$}{若~$a\parallel b$, $a\subset\alpha$, $b\subset\beta$,则~$\alpha\parallel\beta$}{若~$a\parallel \alpha$,
$b\subset\beta$, 则~$a$、$b$ 异面}{若~$a\perp b$, $a\perp\alpha$,$b\perp\beta$, 则~$\alpha\perp\beta$}

\question 已知集合~$A=\{x\mid \abs{x-1}<3 \}$,
集合~$B=\{y| y=x^2+2x+1,x\in\mathbb{R}\}$, 则~$A\cap
\complement_U B$~为\key{C}.
\fourchoices{$[\,0,4)$}{$(-\infty,-2\,]\cup[4,+\infty)$}{$(-2,0)$}{$(0,4)$}

\question 若~$a$、$b$~是直线, $\alpha$、$\beta$~是平面,
则以下命题中真命题是\key{D}.\\
\fourchoices{若~$a$、$b$~异面, $a\subset\alpha$,$b\subset\beta$, 且~$a\perp b$, 则~$\alpha\perp\beta$}{若~$a\parallel b$, $a\subset\alpha$, $b\subset\beta$,则~$\alpha\parallel\beta$}{若~$a\parallel \alpha$,
$b\subset\beta$, 则~$a$、$b$ 异面}{若~$a\perp b$, $a\perp\alpha$,$b\perp\beta$, 则~$\alpha\perp\beta$}

\newpage
\jianda
\question 已知复数~$z$ 满足:$\abs{z}-z^*=\dfrac{10}{1-w\textbf{i}}$(其中~$z^*$
是~$z$ 的共轭复数).
\begin{parts}
\part[7] 求复数~$z$;
\part[7] 若复数~$w=\cos\theta+\textbf{i}\sin\theta\,(\theta\in\mathbb{R})$, 求~$\abs{z-2}$ 的取值范围.
\end{parts}

\begin{solution}
\begin{parts}
\part $z=3+4\textbf{i}$
\part $\abs{z-w}\in[4,6]$
\end{parts}
\end{solution}

\question[14] 函数~$f(x)=4\sin\dfrac{\pi}{12}x\cdot\sin
    \left(\dfrac{\pi}{2}+\dfrac{\pi}{12}x\right),x\in[a,a+1]$,
    其中常数~$a\in[0,5]$, 求函数~$f(x)$ 的最大值~$g(a)$.

\begin{solution}
略
\end{solution}

\newpage

\question[16] 函数~$f(x)=4\sin\dfrac{\pi}{12}x\cdot\sin
    \left(\dfrac{\pi}{2}+\dfrac{\pi}{12}x\right),x\in[a,a+1]$,
    其中常数~$a\in[0,5]$, 求函数~$f(x)$ 的最大值~$g(a)$.

\begin{solution}
略
\end{solution}

\newpage
\question 已知复数~$z$ 满足:$\abs{z}-z^*=\dfrac{10}{1-w\textbf{i}}$(其中~$z^*$
是~$z$ 的共轭复数).
\begin{parts}
\part[8] 求复数~$z$;
\part[8] 若复数~$w=\cos\theta+\textbf{i}\sin\theta\,(\theta\in\mathbb{R})$, 求~$\abs{z-2}$ 的取值范围.
\end{parts}

\begin{solution}
\begin{parts}
\part $z=3+4\textbf{i}$
\part $\abs{z-w}\in[4,6]$
\end{parts}
\end{solution}

\newpage

\question[18] 函数~$f(x)=4\sin\dfrac{\pi}{12}x\cdot\sin
    \left(\dfrac{\pi}{2}+\dfrac{\pi}{12}x\right),x\in[a,a+1]$,
    其中常数~$a\in[0,5]$, 求函数~$f(x)$ 的最大值~$g(a)$.

\begin{solution}
略
\end{solution}

\end{questions}
\end{document}
